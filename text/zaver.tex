\chapter*{Conclusion}
\phantomsection
\addcontentsline{toc}{chapter}{Conclusion}

The main goal of this work was to study the~DAS system and its output files in~the~HDF5 file format. As for the~implementation part, the~goal was to create a~multi-platform application capable of displaying the~data from the~DAS interrogator, processing it, and displaying it in a~suitable form. The visualization should be able to play and pause the~animation of the~flowing data, also with the~ability to convert the~data into an audio file.  

In the~theoretical part, the~principles of optical reflectometry were explained, as well as different types of light scattering - Mie scattering, Rayleigh scattering, Raman scattering, and Brillouin scattering. the~inner workings of the~DAS system and methods like \ac{generalotdr} and \ac{otdr} were also studied. The output file format HDF5 was carefully examined with objects such as Datasets, Groups, Attributes, and Links. The output file structure of the~DAS system OptaSense ODH-F was shown in a~readable form. The existing technology was studied with available software for the~front-end and server side. Real-time capabilities were also discussed with the~data bandwidth requirements. A~prototype was created and written in~Svelte to showcase the~application's design with a~waterfall graph and HTML elements to set properties for display. 

In the~design section, we discuss software requirements and the~basic back-end technologies used later in the~implementation section, such as WebSockets and reading HDF5 files. The front-end technologies for data visualization based on Canvas and SVG graphics were discussed. The basics of the~Svelte framework were introduced.

A web application was created with a~server written in Python and a~client using the~Svelte framework. The back end can read and process the~data in HDF5 file format. The back-end communicates with the~client side using WebSockets. A~simple message system was created for this purpose. The client allows the~user to choose the~file and the~dataset, as well as properties like speed choosing a~colormap and channels to display. The visualization can be started, stopped, and replayed. There is also a~feature to download the~image currently on display.

% In the~last section the~software design of a~real-time application for the~DAS system was discussed. 


% The work can be used as a~resource for creating an application for data visualization in the~browser. The next step is to create fully functional web application - with its server-side written in Python sending data using WebSockets to the~front-end client application written in Svelte.

The application in this state is a~one-page website; the~next step would be to incorporate it into the~SvelteKit framework, allowing page routing. This should be pretty straightforward. Future work can include further improving the~visualization, including zooming and data selection features, either by creating custom \texttt{<canvas>} rendering, SVG rendering with D3js, or using existing libraries such as Plotly. It~might be useful to have multiple algorithms to choose from when processing the~raw DAS data. Although we tried a~few methods for data processing, we have chosen one, and there is no selection possible at this stage.





% \section{}

% The semestral part of the~thesis studies DAS principles

% and the output format of the deployed system, 

% \section{Implementation}

% followed by an implementation of conversion from data to audio signal (in WAV format). 

% \subsection{Real-time possibilities}

% This part also investigates the possibilities of real-time data analysis



% Práce je zaměřena na analýzu dat z optického distribuovaného akustického senzorického (DAS) systému ve formátu HDF5. V semestrální práci student prostuduje princip DAS systém a~jeho výstupy (HDF5 archivy) s detailním popisem formátu HDF5. Následně navrhne program pro převod dat do akustické podoby (WAV) a~popíše možnosti realizace programu v reálném čase. V diplomové práci bude navrženo grafické rozhraní zobrazující data v čase s možnostmi úprav os grafu. Výsledek bude otestován s reálným DAS systémem a~fakultním optickým senzorickým polygonem.




% Info o systeme co pouzivaju v tyme
% 	HW info
% 	outputformat - popisat atributy, kanaly, a~celkovo ako to uklada to toho hdf5ky




% Obsah:
% ako to funguje
% existujuci HW
