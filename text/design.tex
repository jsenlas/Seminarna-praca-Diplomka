
\chapter{Software design}\label{txt.design.design}

There are many ways to implement data visualization. Still, choosing the right solution, programming language, or framework is hard, so this chapter first provides information on what this application should do. Second, it studies the existing OptaSense software and other solutions accessible from the internet. Lastly, it explains the software design decisions for implementing this data visualization.

\section{Software design analysis}\label{txt.design.sw}

\ac{uml} is a~design language made to make it easier for developers and system designers to communicate with each other. \ac{uml} uses different visualization types, graphs, and charts to ease the understanding of complex systems. This way, software design is more understandable and standardized. And when developers, software designers, and management speak the same language - \ac{uml} language - making the development more efficient. It makes management easier as they do not have to understand the implementation but rather understand the meaning of each part of the system. This way, they can make better time assumptions and time planning more predictable. Software designers can specify many diagrams to explain their ideas to developers to showcase the functionality. Also, the diagrams are far more understandable than pure code or pseudo-code implementation. 

There are many different \ac{uml} diagrams. The basic division is to \textit{structural} and \textit{behavioral} diagrams. For this work, we will need three types of diagrams. From the structural diagrams, we will use \textit{Use Case Diagram} and \textit{Class Diagram}. From the behavioral diagrams, we use \textit{Sequence diagram}~\cite{usecase}.

\subsection{Use cases}\label{txt.design.sw.usecase}

A \textit{Use Case Diagram} is an \ac{uml} diagram to form the system or software requirements for a~new software program. It is part of \textit{behavioral diagrams} in the \ac{uml} language. Use case diagrams show the user's point of view of the system, the way they will use the system, with its main functions and features. It consists of the following:

\begin{itemize}
    \item \texttt{Actors} - interacting with software functions; a~noun names them; the actor triggers the use cases.
    \item \texttt{Use case} - the features and functions of the system; are specified by a~phrase describing the action. 
\end{itemize}

Each of the use cases has to have an actor linked to itself. There are also special relationships that extend and include that are not in the interest of this work but are important in general for Use Case diagrams~\cite{usecase}.

The task is to fulfill the usage requirements, as seen in the use case Figure~\ref{fig:usecase}. Users need to see and view what is happening in their perimeter on their screen. For this purpose, the best data visualization is a~waterfall graph, similar to a~spectrogram, displayed as the main element. It should have an editable color map to adjust the sensitivity. The waterfall view should be an animated waterfall graph and ideally display real-time data on the screen, similar to the OS6 system; see Section~\ref{fig:ossix}. In addition, the user should perform selection and zoom on the waterfall graph. The user should be able to edit properties of the graph, like changing the data range and choosing the channels he or she wants to see. The user should be able to export the data to a~WAV file by clicking a~button and then viewing and playing the audio file. There should also be a~waveform display available to show the playing data.

\begin{figure}[h]
    \centering
    \includegraphics{pdf/usecase.drawio.pdf}
    \caption{Application use cases.}
    \label{fig:usecase}
\end{figure}

\subsection{Application requirements}\label{txt.design.sw.requirements}

From the use cases in Section~\ref{txt.design.sw.usecase}, it is understandable that the application should have certain features. Apart from the given use cases, there are other important requirements:

\begin{itemize}
    % \item Support for real-time data plotting - graph updates or animation; fetching the data online from the OptaSense Interrogator and displaying it in real-time
    \item Multiplatform - the application should run on any device and still support all features
    \item Data processing - subsampling data to save data throughput
    \item Plot editing and animation - changing plot properties
    \item Reading offline data - possibility to read local files or upload files into the application
    % \item 
\end{itemize}

% There is the option to create an application running on the operating system

\begin{figure}[h]
    \centering
    \includegraphics{pdf/simple_application.drawio.pdf}
    \caption{Data flow in the application - reading the data, then processing it and displaying it (optionally) edit the view.}
    \label{fig:dataflow}
\end{figure}

It is necessary to choose the correct programming tools to satisfy all features. The right way to find the right solution in programming is to \textit{divide and conquer}. This means finding all the pieces that will make the application. First, there must be an idea of what will happen with the data. Firstly, it has to be read, processed, and then displayed. The optional step is to edit the data or change the view; this data flow can be seen in Figure~\ref{fig:dataflow}. 

% Multiplatform requirement removes many options - like creating a~simple GUI application, as creating a~multiplatform high-performance application is quite a~feat and is beyond this project. So the result is making a~website and display  there is one solution either using 

From the data flow application, an overview can be made for a~web application, as seen in Figure~\ref{fig:app_overview1}. The overview illustrates what parts the application will have. The back-end or server application will be responsible for reading HDF5 files and processing data. It will provide some interface or API for the client application to fetch\footnote{\url{https://developer.mozilla.org/en-US/docs/Web/API/Fetch\_API}} the data. On the client side, the client has to be able to create visualizations of the data and provide a~user interface to change application properties.

\begin{figure}
    \centering
    \includegraphics{pdf/appstack_general.drawio.pdf}
    \caption{Application overview. The server reads the data from the data storage and sends it to the client application, where it is shown to the user.}
    \label{fig:app_overview1}
\end{figure}


\section{Back-end}\label{txt.design.backend}

The purpose of the back-end part of the application is to read, process, and send the data to the client part of the application, as shown in Figure~\ref{fig:app_overview}. Reading the HDF5 data is done as explained in Section~\ref{txt.implementation.reading}.


\subsection{Existing REST based servers for HDF5 data access}\label{txt.design.rest}

\ac{rest} is a~software architecture. More specific is a~RESTful API (\ac{api}), which transfers data between systems on the internet. One of the most common usages is in web services. \ac{rest} uses stateless messages to get data from some resource. The data can be delivered in many different formats. Very popular are \ac{json}, \ac{html}. The communication is based on \ac{http} protocol activities:

\begin{itemize}
    \item \texttt{GET} - request a~record from the other side,
    \item \texttt{POST} - request to create a~record,
    \item \texttt{PUT} - request to update a~record,
    \item \texttt{DELETE} - request to delete a~record~\cite{restapi}
\end{itemize}


There is a~wide range of \ac{rest} server implementations; the HDF Group provides documentation for its \ac{rest}ful API\footnote{\url{https://support.hdfgroup.org/pubs/papers/RESTful\_HDF5.pdf}}. They have prepared a~few RESTful server implementations for their data format. Many more implementations of \ac{rest} servers, such as the Python Simple HTTP server. Here is a~list of some possible implementations~\cite{hdfrest}:

\begin{itemize}
    \item \emph{h5serv} - \ac{rest}-based service to access HDF5 data written in Python (HDF5 Group).
    \item \emph{HSDS} (Highly Scalable Data Service)\footnote{\url{https://github.com/HDFGroup/hsds}} - Python implementation of the \ac{rest}-based service to access HDF5 data stores. Data can be stored in the POSIX file system or object-based storage such as AWS S3. It can run in a~Docker as a~single machine or on a~Kubernetes cluster.
    \item \emph{hdf-rest-api}\footnote{https://github.com/HDFGroup/hdf-rest-api} - is HDF5 \ac{rest} API that provides CRUD support (create, read, update, and delete) for all HDF5 objects.
    \item \emph{h5grove} - Back-end service written in Python providing access to HDF5 file content.
    \item \emph{http.server} - Python implementation of a~simple HTTP server. However, this is better used only for testing when accessing local files.
\end{itemize}


\subsection{WebSockets}\label{txt.design.websocket}

WebSockets (or WebSocket API) enable two-way communication over \ac{tcp}. IEFT standardized it in RFC6455\footnote{https://www.rfc-editor.org/rfc/rfc6455}. All modern browsers support the WebSocket protocol.

The communication starts with an HTTP-compatible handshake, so only one socket can communicate with the server. There are also other header types available for different uses. The server responds with an HTTP Upgrade, the connection is established, and bidirectional communication can begin. Communication closes when either side closes the connection and starts completing the handshake. The other side responds with a~\textit{Close frame} message, closing the connection. 

The protocol is frame-based, as is HTTP protocol, but simultaneously, it tries to be frame-based as little as possible. Just enough to make sure that it can use the HTTP interface for communication; otherwise, it tries to be as minimalist. The authors of the WebSocket protocol wanted it to be low-level, and as close to TCP as possible~\cite{websock}. WebSockets have an implementation in the Python programming language called \verb|websockets|\footnote{https://pypi.org/project/websockets/}.

\begin{figure}[h]
    \centering
    \includegraphics{pdf/websocket.drawio.pdf}
    \caption{WebSocket handshake, communication, and connection close diagram.}
    \label{fig:websocket}
\end{figure}

\newpage
\subsection{Python asyncio library}

Our application must receive and send messages simultaneously when dealing with client-server communication. There are multiple ways of achieving this behavior. Either use multithreading\footnote{https://docs.python.org/3/library/threading.html} or multiprocessing\footnote{https://docs.python.org/3/library/multiprocessing.html} as all of these solutions solve the same problem - doing multiple things at the same time. There is also an alternative way to receive and send messages asynchronously. Asynchronous programming in Python uses an asynchronous library \textit{asyncio}. All of them are a~viable solution to this problem.

It might be a~better solution to use multithreading in the future. Still, it is unnecessary, as it makes the code more complicated and creates an overhead when sharing data between threads. The asynchronous approach is simple, effective, and sufficient for our usage, so we have decided to use it.

There is not that much code involved, and it is relatively easy to program in, although asynchronous programming is quite tricky to understand. In our testing, this solution was working well; see Section~\ref{txt.implementation.python} and Listings~\ref{code.streamdata}.
%TODO.

The \textit{asyncio} library provides a~set of high-level APIs to run Python coroutines, distribute tasks using queues, perform I/O operations, and synchronize tasks in concurrent mode. It has a~\textit{running loop}, to which the developer can register many different \textit{coroutines}, functionalities, and tasks concurrently. It can manage these by using queues, tasks, and events. The Python implementation of WebSockets also uses the asyncio library. The \textit{event loop} is usually run at the beginning of the application. There can be more the one event loop. 

The event loop can register a~\texttt{Task} classes to run multiple tasks. Although it may seem like the tasks are running in parallel, the reality is that the asyncio has a~\textit{scheduler} - a~logic that decides which task should be run at what time. This way, the tasks seem to run in parallel, but in reality, they run in a~single process or thread, and the tasks are just switching from one to another. This enables one thread to wait for a~message from a~client asynchronously and, at the same time, send messages to the client side.

Asyncio also implements \texttt{Event} class, which can be used for awaiting certain messages or conditions. By calling an \texttt{await event.wait()} the coroutine can wait for another coroutine to execute \texttt{event.set()}, which allows the waiting coroutine to continue execution. To stop execution \texttt{await event.clear()} needs to be called, and on the next \texttt{await event.wait()} the execution will stop.

\section{Frontend}\label{txt.design.frontend}

This section provides an overview of visualization techniques and properties for visualizing data from the DAS interrogator described in the previous chapter. It introduces different technologies that can be used for this data visualization. Firstly it describes the JavaScript framework Svelte which is the heart of the application. We provide an assessment of real-time capabilities.

% \subsubsection{WebGL}\label{txt.design.frontend.webgl}
\subsection{Svelte}\label{txt.design.frontend.svelte}

Svelte is a~component framework written in JavaScript, using a~new approach to building web applications. Instead of looking for differences in virtual DOMs as React does, which is done in the browser and consumes quite a~lot of resources, Svelte does everything at the compilation stage. The compilation output is a~JavaScript file \verb|bundle.js|, which contains all the necessary code to run the web application or, better said, it manipulates the DOM directly. The result is a~fast and reactive web page, it also saves resources, and the code can be run on small devices like handheld devices. Web development is also very enjoyable because compilation does not take long, and the changes can be visible immediately. The structure of a~Svelte component consists of three parts - JavaScript code tag \verb|<script>| a~style tag \verb|<style>| and other HTML elements. 

Svelte does not provide more advanced features like page routing. For this purpose, the Svelte team created SvelteKit, which is a~framework for building web applications and allows page routing. Routing is folder-based - the developer creates a~folder and file structure.

\begin{verbatim}
src/routes/about/+page.svelte <=> /about
src/routes/about.svelte <=> /about
\end{verbatim}

Svelte has been changing and has become a~Vite \footnote{\url{https://vitejs.dev}} plugin. Vite provides fast development experience by running a~development server, in the case of Svelte - Hot Module Replacement (HMR)\footnote{\url{https://vitejs.dev/guide/features.html\#hot-module-replacement}}. This way, every change made during development can be immediately seen in the browser without reloading the page, which makes development much faster and more enjoyable. It is necessary to say that Svelte is still in development, and although it is now at version 3, it may change in the future. 

When fetching the data from the server, it is good practice to move this functionality to a~Svelte Store. From a~programming point of view, a~store is an object with a~\texttt{subscribe()}, \texttt{set()} function. An example of a~WebSocket implementation in Svelte can be the Svelte component library \textit{svelte-websocket-store}\footnote{\url{https://github.com/arlac77/svelte-websocket-store}}.




\subsection{Real-time capabilities}\label{txt.design.frontend.realtime}

The data bandwidth (the amount of data necessary to be sent from the server side to the client side) of the application is the biggest factor. The OptaSense Interrogator can produce quite a~lot of data, but if it is saved in an HDF5 file, as it is compressed, it is quite small. As discussed in Section~\ref{txt.implementation.reading}, a~\qty{10}{\second} file produces around \qty{52}{MB} of data. When this data is transformed into text form, it has only \qty{420}{MB}, and when transformed into JSON, it has \qty{946}{MB} as the data are read at  \qty{10}{kSPS}\footnote{\ac{sps}}. Data will be displayed on display with standard resolution and cannot display \qty{10000}{\ac{sps}} on a~small part of the display. Data processing is necessary for this purpose.

\subsection{Data processing}\label{txt.design.frontend.dataproc}

Sending data in the form of REST requests and responses is possible, but it is really useful only when sending a~small HDF5 file as a~whole, not as a~stream of data. The h5grove\footnote{https://pypi.org/project/h5grove/} back-end provides the ability to read sections of the data. The raw data file is unfortunately not suitable for display as it needs to be processed before the visualization. As the files are quite large, it is far better to run some data processing algorithm, see Section~\ref{txt.design.datapprocessing}. So the data need to be smaller, in a~suitable format, and ideally subsampled as it is not viable to display thousands of samples per second on a~small visualization screen.

Data processing can be done on the server or in the browser. As the browser will be busy redrawing waterfall visualization, it is better to process data on the server side. Server-side preprocessing will also save bandwidth as the data will be significantly filtered and in easier to send type. There is no need to send \textit{float64} values as a~text using \ac{json} and WebSockets or with REST API. 


 % The JSON file is quite large - the original HDF5 file is only \qty{52,7}{MB} and the JSON file is \qty{946,6}{MB}. The dump text file is half the size and provides the same information, but the datasets are harder to understand, but the whole file is only half the size of the JSON file at ``only'' \qty{420}{MB}. The JSON format is much easier to read. The structure of the file is divided into three parts:


% %TODO


\subsection{Software design frontend for DAS data visualization}\label{txt.design.frontend.dataviz}

As we discussed in Section~\ref{txt.design.optasense}, it is possible to create such software to display the data in real-time. The proprietary application from Optasense is a~native Windows application. The requirement for this application was to be able to run on multiple platforms. The chosen platform is the Python back-end for opening files, processing the data, and sending the data to the client application using Svelte. The back-end will process the data as explained in Section~\ref{txt.design.datapprocessing}. The waterfall graph will be an HTML Canvas element that displays the data in real-time, redrawing itself as the data arrive at the browser. For ease of displaying the data in Canvas, D3.js will be used. D3 will, for example, apply a~color map to the correct scale according to the data. The user interface written in Svelte will also have inputs to change the properties of the visualization so that the user can select specific channels from the data, choose subsampling effect properties, cutting the frequency range. The ability to export the data to a~WAV file will also be implemented the same way as done in Section~\ref{lab:hdftowav}.

\begin{figure}[h]
    \centering
    \includegraphics{obrazky/appstack.drawio.pdf}
    \caption{Application overview.}
    \label{fig:app_overview}
\end{figure}


\section{HTML chart rendering}

For in-browser rendering, two main options exist using HTML canvas rendering or using \ac{svg} elements. In this section, we will discuss both their advantages over each other and their drawbacks. In general, Canvas rendering performs better than \ac{svg} rendering. This is because SVG is based on a~\ac{dom} structure\footnote{text-based structure defining objects displayed on the web page. https://developer.mozilla.org/en-US/docs/Web/API/Document\_Object\_Model/Introduction}. It is more suitable for large datasets and graphics-heavy games and animations\footnote{https://www.chartjs.org/docs/latest/\#canvas-rendering}.

\subsection{HTML SVG graphics}

\ac{svg} is an XML-based markup language developed by W3C\footnote{World Wide Web Consortium www.w3.org}. It describes 2D vector graphics in XML text files. SVG supports three types of graphic objects: vector shapes, bitmap images, and text. The main advantage compared to bitmap graphics is that all the elements can be rendered at any size without losing quality. 

The description of graphics elements is the same way as the web page description written in HTML to the final web page. There are tags for different geometrical objects, for example, rectangles, ellipses, lines, and animations. Everything is defined in an XML text file which can be edited, searched, or compressed\footnote{https://developer.mozilla.org/en-US/docs/Web/SVG}.

Data visualization using \ac{svg} graphics results in smooth and sharp visualizations. It also enables interactivity with each displayed object - selecting, hovering, or zooming. Libraries for \ac{svg} manipulation manipulate text based on the data given. \ac{dthree} is a~JavaScript library using HTML, SVG, and CSS to visualize data. It can bind data to the HTML DOM and then apply data-driven transformations. \ac{dthree} allows developers to create custom visualizations as it is not a~charting library but a~set of data visualization tools. It is also the base for other charting libraries that build on the \ac{dthree}'f framework. They include Plotly's JavaScript implementation, C3.js\footnote{https://c3js.org/} or Britecharts\footnote{https://britecharts.github.io/britecharts/}.

\subsection{Canvas graphics}\label{txt.design.canvas}

JavaScript makes drawing to the HTML \verb|<canvas>| element possible with the use of \textit{Canvas API} or \textit{WebGL API}. It can render data visualizations, game graphics, real-time video processing, and animation. WebGL can draw 2D and 3D graphics to the canvas element, but we will not be covering WebGL further. Canvas API focuses on 2D graphics. Libraries using Canvas API that make rendering to canvas easier differ on the use case - EaselJS (web game development), Konva.js (desktop and mobile applications), Chart.js, and many more. Canvas rendering depends on the resolution of the screen. 

The \verb|<canvas>| tag is used for drawing graphics. You can imagine the canvas as a~rectangular area with the starting point at the top left corner with coordinates (0,0) and defined width and height. By default, this area is transparent. When we want to draw to it, we need to call functions from the Canvas API. They define basic shapes and primitives that can be drawn on the canvas. Rectangles and paths are the only primitive shapes that can be drawn to \verb|<canvas>| element, and more complex shapes can be drawn by combining these primitives\footnote{https://developer.mozilla.org/en-US/docs/Web/API/Canvas\_API/Tutorial/Drawing\_shapes}. Basic example of drawing to canvas can be seen in the Listing~\ref{lst:canvas.example}. Usually, the functions define certain shapes filled with color and added to the canvas. Then "clearing" functions remove what is displayed or create transparent areas. Lastly, there are "stroke" functions that create lines. 

% \begin{lstlisting}[frame=single,numbers=right,caption={An example implementation of drawing to 
% canvas.},label=lst:canvas.example,basicstyle=\ttfamily\small, keywordstyle=\color{black}\bfseries\underbar,]

\begin{lstlisting}[style=htmlcssjs,label=lst:canvas.example,caption={D3js x-axis implementation.}]
function draw() {
    //First, we need a~reference to canvas element
    const canvas = document.getElementById("canvas");

    //Next, we need context to draw to
    const ctx = canvas.getContext("2d");

    /**** Drawing  ****/
    
    // drawing a~rectangle to canvas
    ctx.fillRect(10, 10, 100, 100);
}
\end{lstlisting}

The biggest drawback of displaying data with \verb|<canvas>| is its lack of interactivity with displayed elements and poor text rendering capabilities\footnote{https://www.w3schools.com/html/html5\_svg.asp}. The interactions are achievable but usually require "hacky" solutions like hidden canvas or invisible layers. Having more than 1000 objects rendered on screen that can be selected can also cause performance issues. 


\subsubsection{Drawing to canvas and animation in canvas}\label{txt.design.animation}

Creating animations on canvas elements is done in two ways. Calling function \verb|setInterval()| in which a~callback function is given with a~time delay value in milliseconds. This was a~preferred way to create animations until the implementation of the \verb|requestAnimationFrame()| method. It also takes a~callback but without a~time value. This is because it just tells the browser to perform an animation, so the developer doe not need to worry about managing the number of frames per second. It can perform a~redraw up to the frame rate of the screen. So if the user has a~\qty{60}{Hz} screen, it redraws every \qty{16.6}{ms} and respectively, if \qty{120}{Hz} screen, it redraws every \qty{8.3}{ms}. For updating at a~lower FPS than the screen refresh rate, the callback is passed with a~time stamp value of the request. This value can be compared with the time of the previous update, and if the delta is smaller than desired, it will not update the canvas. If the delta is bigger, it will allow the redraw.

\subsection{Heatmap visualization}\label{txt.design.frontend.heatmap}

A \textit{heatmap} is a~graph depicting data values in cells in a~two dimension grid. Usually, the cell shape is made of simple rectangles or squares, but any shape is possible. For example, population data can be visualized on a~map or a~globe. The color of each cell is a~representation of the value. Choosing the right color palette; otherwise, the data can be hard to understand. We discuss color palettes in Section~\ref{txt.design.frontend.colormap}. The colors used in the heatmap indicate the relative values of the data. Depending on the color palette chosen, the result can be a~visualization with darker colors representing higher values and lighter colors representing lower values. Showing a~color scale legend near the graph is good practice for more straightforward data interpretation.

Heatmaps are used in different scenarios as they can make data more interesting or easily understandable. They can show the relationship between two variables on two axes or display data the same way as in tables, but thanks to colors making the table more understandable. They can help to find patterns in the data or locate interest points on maps\footnote{https://chartio.com/learn/charts/heatmap-complete-guide/}.

Heatmaps are implemented in different programming languages and frameworks. Some interesting heatmap implementations are D3js \footnote{https://d3-graph-gallery.com/heatmap}, Matlab\footnote{https://www.mathworks.com/help/matlab/ref/heatmap.html}, Plotly\footnote{https://plotly.com/python/heatmaps/}.

A heatmap visualization using the Svelte framework and canvas element can be done in a~few different ways. As the Svelte store updates, the framework also updates everything that this store is used in or connected to. The easiest way is to create a~reactive statement \texttt{\$: command}; see an example in Listing \ref{code.reactivestatement}. When the data store updates its value every time the client sends a~new message with the data, the reactive command executes. This way, we do not need any algorithms to update the data. It is as simple as data arrives and redraws the canvas, as shown in Listing~\ref{code.reactivestatement}. Although this solution is simple, it lacks the possibility to control the speed in any way.

% \label{code.reactivestatement}
\begin{lstlisting}[caption={Svelte reactive statement for redrawing canvas element.},label=code.reactivestatement]
$: {
    $data_store;
    draw()
}    
\end{lstlisting}

The second way of drawing to a~canvas is to call \texttt{requestAnimationFrame()} every time there is new data and then close the animation. This is, however, too much overhead, so we abandoned this idea. A~better approach would be to have the animation loop started by calling \texttt{requestAnimationFrame()}, and each time new data arrives, the chart redraws. The redraw function needs logic placed at the beginning of it to set conditions for when to redraw the canvas according to speed.

Another solution is to redraw the canvas at a~set interval a~few times per second. We use a~\texttt{setInterval()} function to call an \texttt{increment()} function that would be triggered by a~set amount of time determined by a~speed value; see Listing \ref{code.settingspeed}. It needs to be noted that the use of \texttt{setInterval()} function is discouraged since the arrival of \texttt{requestAnimationFrame()} and should not be preferred.

% \label{code.reactivestatement2}
\begin{lstlisting}[caption={Svelte reactive statement for redrawing canvas element using \texttt{setInterval()}. The reactive statement updates on every \texttt{\$last\_row\_number} store update.},label=code.reactivestatement2]
$: {
    $last_row_number;
    draw()
}
\end{lstlisting}

For ease of programming and control over the speed of the animation, we have chosen the solution using the \texttt{setInterval()} function. 

\subsection{Colormaps for a~heatmap chart}\label{txt.design.frontend.colormap}

Choosing a~suitable colormap is crucial when displaying data to users. The purpose of a~colormap is to understandably represent the data so that it is easy for users to understand what is shown in front of them. A~colormap with equal steps between colors and steps in data is best perceived - called \textit{perceptually uniform colormap}. It was found that perception of color change is best understood by the human brain rather than changes in hue. 
\newpage
There are four basic colormap classes based on their function:

\begin{itemize}
    \item Sequential - is used for ordering. Incremental changes in lightness and saturation of the single color, often with a~single hue, for example, \textit{vidris}, \textit{plasma}, \textit{inferno}, \textit{binary}.
    \item Diverging - is used when values deviate around a~median value. It uses changes in lightness and saturation of two different colors that meet in the middle, for example, \textit{spectral}, \textit{PiYG}, \textit{BrBG}, \textit{seismic}.
    \item Cyclic - should be used for values that wrap around at endpoints. It uses two different colors that meet in the middle, beginning, and end, for example, \textit{twilight}, \textit{hsv}.
    \item Qualitative - used for displaying information without order or relationships. Miscellanneous colours, for example \textit{accent}, \textit{paired}, \textit{pastel1}.
\end{itemize}

This section was based on~\cite{colormap}. The data from the DAS system is sequential, so the best type for this application is one of the sequential colormaps. D3.js has a~plugin library for generating colormaps \textit{d3-scale-chromatic}\footnote{https://github.com/d3/d3-scale-chromatic}.

\begin{figure}[h!]
    \centering
    \includegraphics[width=.7\linewidth]{obrazky/sequential.png}
    \caption{Sequential colormap \cite{colormap}.}
    \label{fig:colormap.seq}
\end{figure}

\begin{figure}[h!]
    \centering
    \includegraphics[width=.7\linewidth]{obrazky/cyclic.png}
    \caption{Examples of cyclic colormaps \cite{colormap}.}
    \label{fig:colormap.cyc}
\end{figure}


\begin{figure}[h!]
    \centering
    \includegraphics[width=.7\linewidth]{obrazky/diverging.png}
    \caption{Examples of diverging colormaps \cite{colormap}.}
    \label{fig:colormap.div}
\end{figure}

\begin{figure}[h!]
    \centering
    \includegraphics[width=.7\linewidth]{obrazky/qualitative.png}
    \caption{Examples of qualitative colormaps \cite{colormap}.}
    \label{fig:colormap.qua}
\end{figure}


\bigskip
\section{Prototype}\label{txt.design.frontend.prototype}

To pitch ideas and show the design, a~prototype was made. This prototype was not a~final working application. The prototype was built with the Svelte framework for its high performance and fast development; see Section~\ref{txt.design.frontend.svelte}. The prototype used Flowbite components\footnote{\url{https://flowbite-svelte.com/}} as they make it easy to create stylized web pages. The web page is divided into Svelte components. The main component is the waterfall graph on the left side of the screen and the control panel on the right side. Layout was done with svelte-layouts\footnote{\url{https://www.npmjs.com/package/svelte-layouts}} package. Moving forward, the \textit{svelte-layouts} package will be dropped as it is clunky in this version. Although it provides basic usage, customization is challenging because of the lack of documentation.

Uploading files into the browser using REST API was also tested. Python's \verb|http.server| was used as the back-end. When fetching files into a~browser from local storage using HTTP, it is necessary to allow Cross-Origin Resource Sharing (CORS) because browsers restrict this feature for security reasons. Some resources like CSS, Web Fonts, and WebGL textures have enabled CORS. For sending HDF5 files to the browser, a~special HTTP header has to be added on the server side. Without this feature, the browser would throw an error into the JavaScript console. CORS is not used for the later stages and implementation because we are not uploading whole HDF5 files. It is much more efficient to process the files in the back-end (server) side of the application.

\begin{figure}[h]
    \centering
    \includegraphics[width=\linewidth]{obrazky/svelte_prototype.png}
    \caption{Prototype of DAS visualization application.}
    \label{fig:prototypesvelte}
\end{figure}
