% Pro sazbu seznamu literatury použijte jednu z následujících možností

%%%%%%%%%%%%%%%%%%%%%%%%%%%%%%%%%%%%%%%%%%%%%%%%%%%%%%%%%%%%%%%%%%%%%%%%%
%1) Seznam citací definovaný přímo pomocí prostředí literatura / thebibliography

\begin{thebibliography}{99}


\bibitem{DVSShanFu}
SHAN, Yuanyuan, Jiayun DONG, Jie ZENG, Siyi FU, Yinsen CAI, Yixin ZHANG, and Xuping ZHANG. A~Broadband Distributed Vibration Sensing System Assisted by a~Distributed Feedback Interferometer. IEEE Photonics Journal [online]. 2018, 10(1), 1-10 [cit. 2023-04-17]. ISSN 1943-0655. Dostupné z: doi:10.1109/JPHOT.2017.2776919

\bibitem{perimeterpolsko}
ZYCZKOWSKI, Marek, Edward M. CARAPEZZA, Mieczyslaw SZUSTAKOWSKI, Norbert PALKA a~Marcin KONDRAT. Fiber optic perimeter protection sensor with intruder localization [online]. In: . 2004-11-30, 71- [cit. 2023-04-18]. Dostupné z: doi:10.1117/12.578186

\bibitem{filka}
FILKA, Miloslav. Optické přenosy informací pro integrovanou výuku VUT a~VŠB-TUO. Brno: electronic, 20114. ISBN 978-80-214-5064-6.

\bibitem{cabling}
WOODWARD, Bill, and Andrew OLIVIERO. Cabling The Complete Guide to Copper and Fiber-Optic Networking. 5th edition. Indianapolis, Indiana: John Wiley, 2014. ISBN 978-1-118-80732-3.

\bibitem{scatteringcenterbook}
WEIK, Martin H. Scattering center. In: WEIK, Martin H. Computer Science and Communications Dictionary [online]. Boston, MA: Springer US, 2001, 2000-11-30, s. 1522-1522 [cit. 2023-04-19]. ISBN 978-0-7923-8425-0. Dostupné z: doi:10.1007/1-4020-0613-6\_16662

\bibitem{sveltedoc}
Svelte documentation. Documentation [online]. 2023 [cit. 2023-04-20]. Dostupné z: https://svelte.dev/docs

\bibitem{chemsens}
POSPÍŠILOVÁ, Marie, Gabriela KUNCOVÁ, and~Josef TRÖGL. Fiber-Optic Chemical Sensors and Fiber-Optic Bio-Sensors. Sensors [online]. 2015, 15(10), 25208-25259 [cit. 2023-04-18]. ISSN 1424-8220. Dostupné z: doi:10.3390/s151025208

\bibitem{WangYu2017RDVM}

WANG, Yu, Baoquan JIN, Yuncai WANG, Dong WANG, Xin LIU, Qing BAI: \emph{Real-Time Distributed Vibration Monitoring System Using $\Phi$-OTDR}. IEEE sensors journal [online]. PISCATAWAY: IEEE, 2017, 17(5), 1333-1341 [cit. 2022-11-22]. ISSN 1530-437X. Accessible at: doi:10.1109/JSEN.2016.2642221

\bibitem{dasKislov}
KISLOV, K. V., and V. V. GRAVIROV: \emph{Distributed Acoustic Sensing: A New Tool or a~New Paradigm. Seismic instruments} [online]. Moscow: Pleiades Publishing, 2022, 58(5), 485-508 [cit. 2022-11-22]. ISSN 0747-9239. Accessible at: doi:10.3103/S0747923922050085

\bibitem{hydrophones}
MATSUMOTO, Hiroyuki, Eiichiro ARAKI, Toshinori KIMURA, et al. Detection of hydroacoustic signals on a fiber-optic submarine cable. Scientific reports [online]. England: Nature Publishing Group, 2021, 11(1), 2797-2797 [cit. 2023-04-22]. ISSN 2045-2322. Dostupné z: doi:10.1038/s41598-021-82093-8

\bibitem{dasseismic}
PARKER, Tom, SHATALIN, Sergey, FARHADIROUSAN Mahmoud: \emph{Distributed Acoustic Sensing – a~new tool for seismic applications}. Earthdoc [online]. [cit. 2022-11-22]. Accessible at \(<\)\url{https://doi.org/10.3997/1365-2397.2013034}\(>\)

\bibitem{bhundred}
MERKLEIN, Moritz, Irina V. KABAKOVA, Atiyeh ZARIFI, and Benjamin J. EGGLETON. 100 years of Brillouin scattering: Historical and future perspectives. Applied Physics Reviews [online]. 2022, 9(4), 41306 [cit. 2023-04-21]. ISSN 1931-9401. Accessible at: doi:10.1063/5.0095488

\bibitem{gabai}
GABAI, Haniel, and AVISHAY Eyal: \emph{On the sensitivity of distributed acoustic sensing.} Optics letters vol. 41,24 (2016): 5648-5651. [online]. [cit. 2022-11-22] doi:10.1364/OL.41.005648

\bibitem{wangprogress}
WANG Z, LU B, YE Q, CAI H.: \emph{Recent Progress in Distributed Fiber Acoustic Sensing with $\Phi$-OTDR}. Sensors (Basel). 2020 Nov 18;20(22):6594. doi: 10.3390/s20226594. PMID: 33218051; PMCID: PMC7698859.

\bibitem{kislov_das_newparadigm}
KISLOV, K. V.,  and V. V. GRAVIROV.: \emph{Distributed Acoustic Sensing: A New Tool or a~New Paradigm. Seismic instruments} [online]. Moscow: Pleiades Publishing, 2022, 58(5), 485-508 [cit. 2022-11-25]. ISSN 0747-9239. Accessible at: doi:10.3103/S0747923922050085

\bibitem{hdf5doc}
\emph{The HDF5 Data Model and File Structure} HDF Group [cit. 2022-12-08]. Accessible at \(<\)\url{https://docs.hdfgroup.org/hdf5/develop/_h5_d_m__u_g.html}\(>\).

\bibitem{hdfrest}
HEBER G.: \emph{RESTful HDF5} The HDF Group [cit. 2022-12-08].
\(<\)\url{https://support.hdfgroup.org/pubs/papers/RESTful_HDF5.pdf}\(>\).

\bibitem{seismic}
HORNMAN, J.C.: \emph{Field trial of seismic recording using distributed acoustic sensing with broadside sensitive fibre-optic cables.} Geophysical Prospecting [online]. 2017, 65(1), 35-46 [cit. 2022-12-06]. ISSN 0016-8025. Accessible at: doi:10.1111/1365-2478.12358

\bibitem{progress}
BAO, Xiaoyi, and Liang CHEN.: \emph{Recent Progress in Distributed Fiber Optic Sensors}. Sensors (Basel, Switzerland) [online]. BASEL: Mdpi, 2012, 12(7), 8601-8639 [cit. 2022-12-08]. ISSN 1424-8220. Accessible at: doi:10.3390/s120708601

\bibitem{dasrayleigh}
PALMIERI, Luca; SCHENATO, Luca. Distributed optical fiber sensing based on Rayleigh scattering. The Open Optics Journal, 2013, 7.1.

\bibitem{ytossix}
OptaSense OS6 Visualization Software Demo. YouTube, uploaded by OptaSense, 22 June 2021, \(<\)\url{https://www.youtube.com/watch?v=6-hj1ySERIA}\(>\).

\bibitem{websock}
Melnikov, A., Fette, I.: \emph{The WebSocket Protocol}. RFC Editor. [online] [cit. 2022-08-12]. Accessible at: \(<\)\url{https://doi.org/10.17487/RFC6455}\(>\).

\bibitem{colormap}
Choosing Colormaps in Matplotlib. Choosing Colormaps in Matplotlib [online]. The Matplotlib development team, c2002–2012 [cit. 2023-04-29]. Dostupné z: https://matplotlib.org/stable/tutorials/colors/colormaps.html


\bibitem{fog}
The fiber-optic gyroscope: Herve Lefevre Artec House, 1993, ISBN 0-89006-537-3, pp. 300, £65. Optics and Laser Technology [online]. Elsevier, 1993, 25(6), 406-406 [cit. 2023-04-22]. ISSN 0030-3992. Dostupné z: doi:10.1016/0030-3992(93)90010-D


\bibitem{cors} 
\emph{Cross-Origin Resource Sharing (CORS)} mozilla.org [cit. 2022-12-08]. Accessible at \(<\)\url{https://developer.mozilla.org/en-US/docs/Web/HTTP/CORS}\(>\).

\bibitem{scattering.comparison}
Illustrative spectra of light scattering in optic fibers. Springer Link [online]: Springer [cit. 2023-04-25]. Dostupné z: https://link.springer.com/article/10.3103/S0747923922050085/figures/6

% @misc{rfc6455,
%     series =    {Request for Comments},
%     number =    6455,
%     howpublished =  {RFC 6455},
%     publisher = {RFC Editor},
%     doi =       {10.17487/RFC6455},
%     url =       {https://www.rfc-editor.org/info/rfc6455},
%         author =    {Alexey Melnikov and Ian Fette},
%     title =     {{The WebSocket Protocol}},
%     pagetotal = 71,
%     year =      2011,
%     month =     dec,
%     abstract =  {The WebSocket Protocol enables two-way communication between a client running untrusted code in a controlled environment to a remote host that has opted-in to communications from that code. The security model used for this is the origin-based security model commonly used by web browsers. The protocol consists of an opening handshake followed by basic message framing, layered over TCP. The goal of this technology is to provide a mechanism for browser-based applications that need two-way communication with servers that does not rely on opening multiple HTTP connections (e.g., using XMLHttpRequest or \textless{}iframe\textgreater{}s and long polling). {[}STANDARDS-TRACK{]}},
% }





% \bibitem{sr02/2009}
% 		VUT v~Brně:
%     \emph{Úprava, odevzdávání a zveřejňování vysokoškolských kva\-li\-fi\-kač\-ních prací na VUT v~Brně}\/ [online].
% 		Směrnice rektora č.\,2/2009.
% 		Brno: 2009, po\-sled\-ní aktualizace 24.\,3.\,2009 [cit.\,23.\,10.\,2015].
%     Dostupné z~URL:\\
%     <\url{https://www.vutbr.cz/uredni-deska/vnitrni-predpisy-a-dokumenty/smernice-rektora-f34920/}>.

% \bibitem{CSN_ISO_690-2011}
%     \emph{ČSN ISO 690 (01 0197) Informace a dokumentace -- Pravidla pro bibliografické odkazy a citace informačních zdrojů.}
%     40 stran. Praha: Český normalizační institut, 2011.

% \bibitem{CSN_ISO_7144-1997}
%     \emph{ČSN ISO 7144 (010161) Dokumentace -- Formální úprava disertací a podobných dokumentů.}
%     24 stran. Praha: Český normalizační institut, 1997.

% \bibitem{CSN_ISO_31-11}
%     \emph{ČSN ISO 31-11 Veličiny a jednotky -- část 11: Matematické znaky a značky používané ve fyzikálních vědách a v~technice.}
%     Praha: Český normalizační institut, 1999.

% \bibitem{BiernatovaSkupa2011:CSNISO690komentar}
%     BIERNÁTOVÁ, O., SKŮPA, J.:
%     \emph{Bibliografické odkazy a citace dokumentů dle ČSN ISO 690 (01 0197) platné od 1.\,dubna 2011}\/ [online].
%     2011, poslední aktualizace 2.\,9.\,2011 [cit. 19.\,10.\,2011].
%     Dostupné z~URL:
%     \(<\)\url{http://www.citace.com/CSN-ISO-690.pdf}\(>\)
%    \(<\)\href{http://www.boldis.cz/citace/citace.html}{http://www.boldis.cz/citace/citace.html}\(>\).

% \bibitem{pravidla}
%     \emph{Pravidla českého pravopisu}.
%     Zpracoval kolektiv autorů. 1.\ vydání.
%     Olomouc: FIN PUB\-LISH\-ING, 1998. 575 s. ISBN 80-86002-40-3.

% \bibitem{Walter1999}
% 	WALTER, G.\,G.; SHEN, X.
% 	\emph{Wavelets and Other Orthogonal Systems}.
% 	2. vyd. Boca Raton: Chapman\,\&\,Hall/CRC, 2000. 392~s. ISBN 1-58488-227-1

% \bibitem{Svacina1999IEEE}
% 	SVAČINA, J.
% 	Dispersion Characteristics of Multilayered Slotlines -- a Simple Approach.
% 	\emph{IEEE Transactions on Microwave Theory and Techniques},
% 	1999, vol.\,47, no.\,9, s.\,1826--1829. ISSN 0018-9480.

% \bibitem{RajmicSysel2002}
%     RAJMIC, P.; SYSEL, P.
%     Wavelet Spectrum Thresholding Rules.
%     In \emph{Proceedings of the International Conference Research in Telecommunication Technology},
%     Žilina: Žilina University, 2002. s.\,60--63. ISBN 80-7100-991-1.

\end{thebibliography}


%%%%%%%%%%%%%%%%%%%%%%%%%%%%%%%%%%%%%%%%%%%%%%%%%%%%%%%%%%%%%%%%%%%%%%%%%
%%2) Seznam citací pomocí BibTeXu
%% Při použití je nutné v TeXnicCenter ve výstupním profilu aktivovat spouštění BibTeXu po překladu.
%% Definice stylu seznamu
%\bibliographystyle{unsrturl}
%% Pro českou sazbu lze použít styl czechiso.bst ze stránek
%% http://www.fit.vutbr.cz/~martinek/latex/czechiso.tar.gz
%%\bibliographystyle{czechiso}
%% Vložení souboru se seznamem citací
%\bibliography{text/literatura}
%
%% Následující příkaz je pouze pro ukázku sazby literatury při použití BibTeXu.
%% Způsobí citaci všech zdrojů v souboru literatura.bib, i když nejsou citovány v textu.
%\nocite{*}